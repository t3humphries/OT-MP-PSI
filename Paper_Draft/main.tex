\documentclass[10pt, sigconf]{acmart}
\settopmatter{printacmref=false}
\renewcommand\footnotetextcopyrightpermission[1]{} 
\usepackage[utf8]{inputenc}
\usepackage{comment}
\usepackage{multirow}
\usepackage{pifont}
\usepackage{xifthen}
\usepackage[underline=false]{pgf-umlsd}
\usepackage{tikz}
\usepackage{algorithm}
\usepackage[noend]{algpseudocode}
\usepackage{algpseudocode}
\newcommand{\formatcomment}[1]{\scriptsize\textcolor{blue!25!black}{\texttt{#1}}}
\newcommand{\InlineComment}[1]{\vspace*{0.5ex}\State\formatcomment{/*\,#1\,*/}\vspace*{0.5ex}}
\algrenewcommand{\algorithmiccomment}[1]{\hfill\parbox{5.2cm}{\formatcomment{//\,#1}}}
\algrenewcommand\algorithmicprocedure{\textbf{algorithm}}
\algnewcommand\algorithmicforeach{\textbf{for each}}
\algdef{S}[FOR]{ForEach}[1]{\algorithmicforeach\ #1\ \algorithmicdo}
\usepackage{enumitem}
\usepackage[lambda ,advantage, operators, sets , adversary, landau, probability, notions, logic, ff , mm, primitives, events, complexity, asymptotics, keys]{cryptocode}
\usepackage{booktabs} % For formal tables
%\usepackage{color,soul}
\usepackage{comment}
\usepackage{amssymb}
\usepackage[normalem]{ulem} % use normalem to protect \emph
 \newcommand{\superscript}[1]{\ensuremath{{}^{\textrm{\scriptsize #1}}}}
 \newcommand{\roughtext}[1]{\begin{color}{SkyBlue}#1\end{color}}
 \newcommand{\mntext}[1]{\colorbox{SkyBlue}{\begin{color}{black}#1\end{color}}}
 \newcommand{\mn}[2][]{{\tiny\superscript{\mntext{\arabic{mn}}}}\marginpar{\scriptsize{
 			\ifthenelse{\isempty{#1}}
 			{\mntext{\parbox{0.95\marginparwidth}{\superscript{\arabic{mn}}~\raggedright{#2}}}}
 			{\mntext{\parbox{0.95\marginparwidth}{\superscript{\arabic{mn}}#1 says~:~\raggedright{#2}}}}
 		}}\stepcounter{mn}}
 \newcommand{\cryptost}{\mathsf{st}}

\newcommand{\bmargin}[2]{\textcolor{purple}{\ensuremath{^{#1}}}\marginpar{\scriptsize \raggedright \textcolor{purple}{\ensuremath{^{#1}}Bailey: #2}}}


\title{Over-Threshold Multi-Party Private Set Intersection}
\date{}
\usepackage{natbib}
\usepackage{graphicx}
\onecolumn

\begin{document}
\pagestyle{plain} % removes running headers
\onecolumn
\begingroup
\centering
{\Huge Over-Threshold Multi-Party Private Set Intersection}\\[3em]
\endgroup
% \maketitle
% \begin{abstract}
% \end{abstract}

\section{Introduction}

Use case: collaborating network operations centers under different legislations

Brief problem intro

Comparison to Kissner, Song protocol \cite{Kissner}: \\
- $O(1)$ rounds vs. $O(m)$

Comparison to generic secure computation: \\
- lower communication complexity $O(nmk)$ vs. $O(nm^3 \log^2 nm)$ -- assuming optimal circuit is $O(nm \log^2 nm)$ \\
- constant round multi-party protocols not yet practical

SS-OPRF

Two variants: SS in the exponent, communication O(nmk); SS in the base, communication O(nmkt)

Contributions \\
- new construction of over-threshold multi-party private set intersection protocol with communication ... and security ... \\
- two secret-shared oblivious pseudo-random function (SS-OPRF) \\
- practical evaluation?

\section{Problem Description}

$m$ parties, each with a set of size at most $n$, would like to compute the set of elements that occur in at least $t$ participants' sets and who the corresponding owners are without revealing elements outside of the threshold intersecting set


\section{Background}
\subsection{OPRF}

An oblivious pseudo-random function (OPRF) \cite{OPRF,Pinkas} is
a two-party protocol where the client can query a pseudo-random function
$F_{k}$ on input values $x$, in such a way that the key $k$, generated
by the server, remains hidden from the client, while at the same time
the client's input $x$ is not revealed to the server. OPRFs can be
established through generic methods for secure multiparty computation
(on top of circuits that implement ordinary pseudo-random functions),
or by means of the Diffie-Hellman assumption. For instance, the client
chooses a random $r$ and sends $x^{r}$ to the server; the server
responds by sending back $x^{kr}$, from which the client can obtain
$x^{k}=(x^{kr})^{-r}$.

\subsection{Secret Sharing}

In secret sharing, the goal is to distribute among $m$ parties \emph{shares
}$s_{1},\ldots,s_{m}$ of a secret $s$. An \emph{$(m,t)$-threshold
scheme} is such all $t$ of the shares are sufficient to \emph{reconstruct
}$s$ but no group of parties possessing fewer than $t$ shares can
infer any information about $s$. In Shamir's secret sharing scheme
\cite{Shamir}, the distributing party forms the polynomial 
\[
f(x)=a_{t-1}x^{t-1}+a_{t-2}x^{t-2}+\ldots a_{1}x+s
\]
over a finite field of prime order, with $a_{1},\ldots,a_{t-1}$ randomly
generated. It then produces the shares by evaluating $f$ at $m$
publicly-known distinct values: for instance, by setting $s_{i}=(i,f(i))$.
Since $t$ points uniquely determine a polynomial of degree $t$,
given all the $s_{i}$, anyone possessing $t$ shares can use Lagrange
interpolation to recover $f$ and thus obtain $s$. However, fewer
than $t$ shares together reveal no information about $s$.

\subsection{SS-OPRFs}

OPRF that outputs secret share of PRF

Note that the ss polynomial needs to be different for every message

\subsection{Paillier and D\aa mgard Jurik Crytposystems}

The Paillier cryptosystem \cite{Paillier} is an additively homomorphic
scheme for public key encryption based on the intractability of the
Composite Residuosity Class Problem. Ciphertexts are regarded as elements
over the multiplicative group $\mathbb{Z}_{n^{2}}^{*}$, where $n=pq$
for primes $p$ and $q$. Letting $\lambda=\text{lcm}(p-1,q-1)$,
and choosing $g$ to be a random base such that its order is divisible
by $n$, the public key is $(n,g)$ and the private key is $(p,q)$.
Encrypting a plaintext message $m<n$ is done by choosing a random
$r<n$ and computing: 
\[
Enc(m)=g^{m}r^{n}\mod n^{2}.
\]
A given ciphertext $c<n^{2}$ is decrypted using $\lambda$: 
\[
Dec(c)=\frac{L(c^{\lambda}\text{ mod }n^{2})}{L(g^{\lambda}\text{ mod }n^{2})}\mod n,
\]
where 
\[
L(x)=\frac{x-1}{n}.
\]
Such a scheme satisfies the following homomorphic properties (the
moduli are implicit): 
\begin{align*}
Dec(Enc(m_{1})Enc(m_{2})) & =m_{1}+m_{2}\\
Dec(Enc(m)^{k}) & =km.
\end{align*}

D\aa mgard Jurik \cite{Damgard} 

\section{Protocol Overview}
% From http://tex.stackexchange.com/questions/164707/how-to-use-greek-letters-in-pgf-umlsd-or-generally-terms-starting-with
\renewcommand{\mess}[4][0]{
  \stepcounter{seqlevel}
  \path
  (#2)+(0,-\theseqlevel*\unitfactor-0.7*\unitfactor) node (mess from) {};
  \addtocounter{seqlevel}{#1}
  \path
  (#4)+(0,-\theseqlevel*\unitfactor-0.7*\unitfactor) node (mess to) {};
  \draw[->,>=angle 60] (mess from) -- (mess to) node[midway, above]
  {#3};
  \node (\detokenize{#3} from) at (mess from) {};
  \node (\detokenize{#3} to) at (mess to) {};
}

% From http://tex.stackexchange.com/questions/98525/pgf-umlsd-and-externalize
\newcommand{\sdinit}{%
   \pgfdeclarelayer{umlsd@background}%
   \pgfdeclarelayer{umlsd@threadlayer}%
   \pgfsetlayers{umlsd@background,umlsd@threadlayer,main}%
}
\newcommand{\sdbegin}{%
   \setlength{\unitlength}{1cm}%
   \tikzstyle{sequence}=[coordinate]%
   \tikzstyle{inststyle}=[rectangle, draw, anchor=west, minimum
   height=0.8cm, minimum width=1.6cm, fill=white, 
   drop shadow={opacity=1,fill=black}]%
   \ifpgfumlsdroundedcorners%
      \tikzstyle{inststyle}+=[rounded corners=3mm]%
   \fi%
   \tikzstyle{blockstyle}=[anchor=north west]%
   \tikzstyle{blockcommentstyle}=[anchor=north west, font=\small]%
   \tikzstyle{dot}=[inner sep=0pt,fill=black,circle,minimum size=0.2pt]%
   \global\def\unitfactor{0.6}%
   \global\def\threadbias{center}%
   % reset counters
   \setcounter{preinst}{0}%
   \setcounter{instnum}{0}%
   \setcounter{threadnum}{0}%
   \setcounter{seqlevel}{0}%
   \setcounter{callevel}{0}%
   \setcounter{callselflevel}{0}%
   \setcounter{blocklevel}{0}%
   % origin
   \node[coordinate] (inst0) {};%
}
\newcommand{\sdend}{%
   \begin{pgfonlayer}{umlsd@background}%
      \ifnum\value{instnum}>0%
         \foreach \t [evaluate=\t] in {1,...,\theinstnum}{%
            \draw[dotted] (inst\t) -- ++(0,-\theseqlevel*\unitfactor-2.2*\unitfactor);%
         }%
         
      \fi%
      \ifnum\value{threadnum}>0%
         \foreach \t [evaluate=\t] in {1,...,\thethreadnum}{%
            \path (thread\t)+(0,-\theseqlevel*\unitfactor-0.1*\unitfactor) node (threadend) {};%
            \tikzstyle{threadstyle}+=[threadcolor\t]%
            \drawthread{thread\t}{threadend}%
         }%
      \fi%
   \end{pgfonlayer}%
}

\subsection{Notations and Assumptions}
\begin{itemize}
    \item m = the number of parties each identified by $id_i$; $i = 1, \cdots, m$
    \item n = max size of the set of elements, $\mathbb{L}_i$, owned by $id_i$; $i = 1, \cdots, m$
    \item t = threshold (intersection and the secret sharing threshold)
    \item b = number of bins; $b_1, \cdots, b_b$
    \item Each bin is padded to max
    \item Key holder is semi-honest
    \item Key holder is not a party
    \item Key holder sees at most t-1 shares
\end{itemize}
\subsection{Scheme 1}
Set up: field $\mathbb{F}$ with prime p, generator g, hash functions $H(\cdot)$ is used in share generation and $H_B(\cdot)$ is used for hashing-to-bins, global randomnesses: i) key holder: $K_1$ and $K_2$; $K_1$ is just known by key holder, but $k_2$ is known by all, ii) the random numbers $r_1, \cdots, r_{t-1} \gets \mathbb{F}_p$ generated by the key holder, that are fixed for all participants and their all elements. Participants use a homomorphic encryption scheme, Enc, with pubic and private keys $pk, sk$.
\subsubsection{Share Generation}\label{ShareGen_S1}
The key holder generates random numbers $r_1, \cdots, r_{t-1} \gets \mathbb{F}_p$. Then it communicates with each participant to generate shares for each participant's element using Shamir secret sharing scheme. Figure \ref{fig:S1_ShareGen} shows the protocol taking place between the key holder and a participant with identifier identifier $id_i$, $i =1, \cdots, m$, to generate shares for an element $l$. This protocol iterates over all elements $l \in \mathbb{L}$ to generate the corresponding share for each element owned by $id_i$.
\begin{figure}[h!]
   \centering
   \sdinit{}
   \begin{tikzpicture}
      % Define symbols and names for the parties
      \sdbegin{}
      \newinst{A}{Key holder}
      \newinst[5]{B}{Participant} % Increase "5" to widen
      
    % Message from B to A, with computations by both sides
      \postlevel
      \mess{B}{$H(l)^\alpha, g^\alpha $}{A}
      \node[anchor=west] at (mess from) {\shortstack[c]{
            element \emph{l}, ${id_i}$ \\
            $\alpha \gets \mathbb{F}_p$}};
      \node[anchor=east] at (mess to) {$H(l)^{K_1 \alpha}$};
      
      % Message from A to B, with precomputations
      \postlevel
      \mess{A}{$g^{r \alpha}  H(l)^{K_1 \alpha}$}{B}
      \node[anchor=east] at (mess from) {$r \gets \mathbb{F}_p$};
      \node[anchor=west] at (mess to) {$g^{r} H(l)^{K_1}$};

      \postlevel
      \mess{B}{$Enc_{pk}[g^r H(l)^{K_1}]$}{A}
      \node[anchor=west] at (mess from) {\shortstack[l]{
            Enc: homomorphic \\encryption scheme}};
      \node[anchor=east] at (mess to) {$Enc_{pk}[H(l)^{K_1}]$};

      \postlevel
      \postlevel
      \mess{A}{\shortstack[l]{$\{g^{r \alpha} H(l)^{r_j \alpha}\}_{j = 1, \cdots, t-1}$\\$\{g^{r \alpha} H(l)^{r'_j \alpha}\}_{j = 1, \cdots, t-1}$ }}{B}
      \node[anchor=east] at (mess from) {\shortstack[l]{
      For global $r_j$s\\
      And global $r'_j$s
      }};
      \node[anchor=west] at (mess to) {\shortstack[l]{
      $\{g^{r} H(l)^{r_j}\}_{j = 1, \cdots, t-1}$\\
      $\{g^{r} H(l)^{r'_j}\}_{j = 1, \cdots, t-1}$
      }}; 
      
      \postlevel
      \postlevel
      \postlevel
      \mess{B}{\shortstack[l]{$\{Enc_{pk}[g^{r}  H(l)^{r_j}]\}_{j = 1, \cdots, t-1}$\\$\{Enc_{pk}[g^{r}  H(l)^{r'_j}]\}_{j = 1, \cdots, t-1}$}}{A}
      \node[anchor=west] at (mess from) {\shortstack[l]{
            Enc: homomorphic \\encryption scheme}};
      \node[anchor=east] at (mess to) {\shortstack[l]{
      $\{Enc_{pk}[H(l)^{r_j}]\}_{j = 1, \cdots, t-1}$\\
      $\{Enc_{pk}[H(l)^{r'_j}]\}_{j = 1, \cdots, t-1}$}};

      \postlevel
      \postlevel
      \postlevel
      \mess{A}{\shortstack[c]{$P(id_i)$\\$P'(id_i)$}}{B}
      \node[anchor=east] at (mess from) {\shortstack[r]{
       $P(x) = \sum_{}x^j Enc[H(l)^{r_j}] $\\
       $ + Enc[H(l)^{K_1}]$
       \vspace{.3em}
       \\$P'(x) = \sum_{}x^j Enc[H(l)^{r'_j}] $\\
       $+ Enc[H(l)^{K_1K_2}]$}};
      \node[anchor=west] at (mess to) {\shortstack[c]{$Share_i(l) = Dec_{sk}[P(id_i)]$\\
      $MAC_i(l) = Dec_{sk}[P'(id_i)]$}};        
      \sdend{}
   \end{tikzpicture}
   \caption{Communication between the key holder and a participant $id_i$ in scheme 1 - Share generation for an element $l \in \mathbb{L}_i$, owned by $id_i$}
   \label{fig:S1_ShareGen}
\end{figure}

\subsubsection{Hashing-to-bins}\label{Binning_S1}
Participants hash their shares to bins to reduce the computation cost for the reconstructor. Corresponding shares to an element $l$ are stored in bin number $H_B(l)$. Each stored value is a quadruple consisting of i) the participant's identifier, $id_i$, ii) $id_i$'s share for its element $l$, $Share_i(l)$, iii) the corresponding MAC, $MAC_i(l)$, and iv) the number of the bin, $H_B(l)$, that stores the information.

\subsubsection{Reconstruction}\label{Recon_S1}
After each participant stored their shares in the corresponding bins, the reconstructor  $\mathcal{R}$ -- who is also a participant -- reconstructs secrets in each bin. For each $m \choose t$ subset of the participants, $id_{i_1}, \cdots, id_{i_t}$, $\mathcal{R}$ collects the shares and reconstructs the corresponding secret by applying Lagrange interpolation: $H(l)^{k_1} = \sum^t_{w=0} Share_{i_w}(\cdot)(\prod_{w' \neq w} \frac{-id_{i_{w'}}}{id_{i_w} - id_{i_{w'}}})$. Reconstruction's steps are summarized in the Algorithm \ref{alg:S1_Recon}. 

\begin{algorithm}[h!]
	 \caption[\textsc{Reconstruct\textsubscript{Scheme1}}]{\textsc{Reconstruct\textsubscript{Scheme1}}}\label{alg:S1_Recon}
	 	\begin{algorithmic}[1]
	 	    \InlineComment{$K_1$ is key holder's secret key and $K_2$ is a publicly known key used for MAC generation}
	 		\InlineComment{$H(\cdot)$ is a hash function in share generation}
	 		\InlineComment{$H_B(\cdot)$ is a hash functions used for hashing-to-bin (Section \ref{Binning_S1})}
            \ForEach {Participant $id_i$;  $i = 1, \cdots, m$}
                \ForEach {Element $l$ owned by $id_i$; $l \in \mathbb{L}_i$}
                    \State $id_i$: store $(id_i, Share_i(l), MAC_i(l), H_B(l))$ in the bin number $H_B(l)$
                    \State
	 	        \EndFor
	 	    \EndFor 
	 	    \ForEach {Bin $b_z$;  $z = 1, \cdots, b$}
	 	        \ForEach {t-subset of quadruples in the bin $b_z$; $\{(id_{i_1}, Share_{i_1}(l), MAC_{i_1}(l), b_z), \cdots, (id_{i_t}, Share_{i_t}(l), MAC_{i_t}(l), b_z)\}$}
	 	            \State $\mathcal{R}$: Apply Lagrange interpolation to  find the corresponding intercept, $Secret_{share}$, for the polynomial that covers the points $\{({i_1}, Share_{i_1}), \cdots, ({i_t}, Share_{i_t})\}$
	 	            \State $\mathcal{R}$: Apply Lagrange interpolation to  find the corresponding intercept, $Secret_{MAC}$, for the polynomial that covers the points $\{({i_1}, MAC_{i_1}), \cdots, ({i_t}, MAC_{i_t})\}$
	 	            \If {$Secret_{MAC} == (Secret_{share})^{K_2}$}{ Reveal that $(id_{i_1}, \cdots , id{i_t})$ can reconstruct the $Secret_{share}$ which is $H(l)^{K_1}$}
	 	            \EndIf 
	 	        \EndFor
	 	    \EndFor
	 	\end{algorithmic}
\end{algorithm}


\subsubsection{Complexity}
\begin{theorem}
The communication complexity of Scheme 1 is $\cdots$
% $O(nm^2)$ or $O(nmtc)$?
\end{theorem}
\begin{proof}
\end{proof}
\begin{theorem}
The computation complexity of Scheme 1 is $\cdots$
%${m \choose t} n \log{n}^m$.
\end{theorem}
\begin{proof}
\end{proof}

\subsection{Scheme 2}
In this scheme, the polynomial in Shamir secret sharing scheme is calculated in the exponent, the result is eliminating the need to use the homomorphic encryption scheme, Enc. This change reduces the communication cost of the scheme as described in Section \ref{ShareGen_S2}. Similar to Scheme 1, we have the following set up parameters: field $\mathbb{F}$ with prime p, generator g, hash functions $H(\cdot)$ is used in share generation and $H_B(\cdot)$ is used for hashing-to-bins, global randomnesses: i) key holder: $K_1$ and $K_2$; $K_1$ is just known by key holder, but $k_2$ is known by all, ii) the random numbers $r_1, \cdots, r_{t-1} \gets \mathbb{F}_p$ generated by the key holder, that are fixed for all participants and their all elements. 

\subsubsection{Share Generation}\label{ShareGen_S2}
\begin{figure}[h!]
   \centering
   \sdinit{}
   \begin{tikzpicture}
      % Define symbols and names for the parties
      \sdbegin{}
      \newinst{A}{Key holder}
      \newinst[5]{B}{Participant} % Increase "5" to widen
      
    % Message from B to A, with computations by both sides
    %   \postlevel
    %   \mess{A}{\shortstack[l]{$\{g^{r_i}\}_{i = 1, \cdots, t-1}$\\$\{g^{r'_i}\}_{i = 1, \cdots, t-1}$ }}{B}
    %   \node[anchor=east] at (mess from) {\shortstack[l]{
    %   For global $r_i$s\\
    %   And global $r'_i$s
    %   }};
    %   \node[anchor=west] at (mess to) {\shortstack[l]{
    %   $\{g^{r_i}\}_{i = 1, \cdots, t-1}$\\
    %   $\{g^{r'_i}\}_{i = 1, \cdots, t-1}$
    %   }}; 
    
      \postlevel
      \postlevel
      \mess{B}{$H(l)^\alpha$}{A}
    %   \mess{B}{\shortstack[l]{$H(l)^\alpha , \{g^{r_i \alpha}\}_{i = 1, \cdots, t-1}$ \\ $H(l)^\alpha , \{g^{r'_i \alpha}\}_{i = 1, \cdots, t-1}$}}{A}      
       \node[anchor=west] at (mess from) {\shortstack[c]{
            element \emph{l}, \emph{id} \\
            $\alpha \gets \mathbb{F}_p$}};
    \node[anchor=east] at (mess to) {\shortstack[l]{$H(l)^{\alpha K_1}, \{H(l)^{r_i \alpha}\}_{i = 1, \cdots, t-1}$ \\ $H(l)^{\alpha K_1 K_2}, \{H(l)^{r'_i \alpha}\}_{i = 1, \cdots, t-1}$}};
    %   \node[anchor=east] at (mess to) {\shortstack[l]{$H(l)^{\alpha K_1}, \{g^{r_i \alpha}\}_{i = 1, \cdots, t-1}$ \\ $H(l)^{\alpha K_1 K_2}, \{g^{r'_i \alpha}\}_{i = 1, \cdots, t-1}$}};
      
      \postlevel
      \postlevel
      \mess{A}{\shortstack[c]{$P(id)$\\$P'(id)$}}{B}
      \node[anchor=east] at (mess from) {\shortstack[r]{
       $P(x) = H(l)^{\alpha \sum_{}r_i x^i} H(l)^{\alpha K_1}$
       \vspace{.4em}
       \\$P'(x) = H(l)^{\alpha \sum_{}r'_i x^i} H(l)^{\alpha K_1 K_2}$}};
      \node[anchor=west] at (mess to) {\shortstack[c]{$Share(l) = P(id)^{1/ \alpha}$\\
      $MAC(l) = P'(id)^{1/ \alpha}$}};        
      
      \sdend{}
   \end{tikzpicture}
   \caption{Communication between the key holder and a participant $id_i$ in scheme 2 - Share generation for an element $l \in \mathbb{L}_i$, owned by $id_i$}
   \label{fig:S2_ShareGen}
\end{figure}
The key holder generates random numbers $r_1, \cdots, r_{t-1} \gets \mathbb{F}_p$. Then it communicates with each participant to generate shares for each participant's element using Shamir secret sharing scheme, by forming a polynomial in the exponent. Figure \ref{fig:S2_ShareGen} shows the protocol taking place between the key holder and a participant with identifier identifier $id_i$, $i =1, \cdots, m$, to generate shares for an element $l$. This protocol iterates over all elements $l \in \mathbb{L}$ to generate the corresponding share for each element owned by $id_i$.

\subsubsection{Hashing-to-bins}\label{Binning_S2} This step is identical to the hashing-to-bins described in Section \ref{Binning_S1}.

\subsubsection{Reconstruction}\label{Recon_S2}
Similar to the reconstruction in Section \ref{Recon_S1} for Scheme 1, each participant stores their shares in the corresponding bins. Then the reconstructor $\mathcal{R}$ --who is also participant-- reconstructs the secrets in each bin for every $m \choose t$ subset of the participants, $id_{i_1}, \cdots, id_{i_t}$, $\mathcal{R}$. As the polynomial is in the exponent of the generator in this scheme, the reconstructor applies the Lagrange interpolation in the exponent to calculate the secret as follows: $H(l)^{k_1} = \prod^t_{w=0} Share_{i_w}(\cdot)^{(\prod_{w' \neq w} \frac{-id_{w'}}{id_w - id_{w'}})}$.

\begin{algorithm}[h!]
	 \caption[\textsc{Reconstruct\textsubscript{Scheme2}}]{\textsc{Reconstruct\textsubscript{Scheme2}}}\label{alg:S2_Recon}
	 	\begin{algorithmic}[1]
	 	    \InlineComment{$K_1$ is key holder's secret key and $K_2$ is a publicly known key used for MAC generation}
	 		\InlineComment{$H(\cdot)$ is a hash function in share generation}
	 		\InlineComment{$H_B(\cdot)$ is a hash functions used for hashing-to-bin (Section \ref{Binning_S1})}
            \ForEach {Participant $id_i$;  $i = 1, \cdots, m$}
                \ForEach {Element $l$ owned by $id_i$; $l \in \mathbb{L}_i$}
                    \State $id_i$: store $(id_i, Share_i(l), MAC_i(l), H_B(l))$ in the bin number $H_B(l)$
                    \State
	 	        \EndFor
	 	    \EndFor 
	 	    \ForEach {Bin $b_z$;  $z = 1, \cdots, b$}
	 	        \ForEach {t-subset of quadruples in the bin $b_z$; $\{(id_{i_1}, Share_{i_1}(l), MAC_{i_1}(l), b_z), \cdots, (id_{i_t}, Share_{i_t}(l), MAC_{i_t}(l), b_z)\}$}
	 	            \State $\mathcal{R}$: Reconstruct the corresponding secret, $Secret_{share}$, to the points $\{({i_1}, Share_{i_1}), \cdots, ({i_t}, Share_{i_t})\}$
	 	            \State $\mathcal{R}$: Reconstruct the corresponding secret, $Secret_{MAC}$, to the points $\{({i_1}, MAC_{i_1}), \cdots, ({i_t}, MAC_{i_t})\}$
	 	            \If {$Secret_{MAC} == (Secret_{share})^{K_2}$}{ Reveal that $(id_{i_1}, \cdots , id{i_t})$ can reconstruct the $Secret_{share}$ which is $H(l)^{K_1}$}
	 	            \EndIf 
	 	        \EndFor
	 	    \EndFor
	 	\end{algorithmic}
\end{algorithm}

\subsubsection{Complexity}
\begin{theorem}
The communication complexity of Scheme 2 is $\cdots$.
\end{theorem}
\begin{proof}
\end{proof}
\begin{theorem}
The computation complexity of Scheme 2 is $\cdots$.
\end{theorem}
\begin{proof}
\end{proof}


\section{Security}
\begin{definition}
We say the Decisional Diffie-Hellman (DDH) Assumptions holds if
$$
\pseudocode{
a, b, c \sample \FF_{\phi(p)}^3 \\
\prob{\adv(g^a, g^b, g^c) = 1} - \prob{\adv(g^a, g^b, g^{ab}) = 1} < \negl \\
}
$$
\end{definition}

\begin{theorem}
\label{thm:sec_prf}
If the DDH Assumption holds and $H$\bmargin{*}{which H. The hash function $H(\cdot)$ or need a different notation here } is a programmable random oracle, the output of $t$ participants in Scheme 2 is a pseudo-random function.
\end{theorem}

\begin{proof}
We prove by reducing an adversary in $\pcgamename^{\prf}$ to an adversary against the DDH Assumption.
Let $g^a, g^b, g^c$ be a DDH instance.
We program $H$ to output $g^a$ on input $x$.
We set the public $\pk$ to $g^b$.
We can still answer queries for $F_{k}(x')$ where $x' \neq x$ by programming $H$ to choose and store $r \sample \FF_{\phi(p)}$ and output $R = g^r$.
An answer to an OPRF query for $x'$ is then $g^{b r} = H(x')^b$.
We set $y_b$ to $g^c$.
We set the output of adversary $\adv^{\mathsf{DDH}}$ to the output of adversary $\adv^{\prf}$.
All outputs are indistinguishable from the real scheme and any advantage between the two adversaries translates directly.
\end{proof}

\begin{theorem}
Scheme 2 is a secret-shared oblivious pseudo-random function.
\end{theorem}

\begin{corollary}
There exist simulators $\simulator_{Participant}(x, Share_{i}(F_{k}(x))$ and $\simulator_{Keyholder}(k)$ that are computationally indistinguishable from the messages received during the protocol.
\end{corollary}

\begin{proof}
Correctness is given by the protocol construction, such that $Share_{i}(F_{k}(x))$ is returned to the participant and $t' \leq t$ participants\bmargin{*}{I'm confused by the leq reconstruction here} can reconstruct $F_k(x)$.

Since each coefficient of the Shamir secret shares is the output of a PRF (Theorem~\ref{thm:ec_prf}), the output of Scheme 2 is computationally indistinguishable from a secret share of $F_{k}(x)$.
Furthermore, $t-1$ secret shares of $F_{k}(x)$ are perfectly indistinguishable from a set of $t-1$ uniformly chosen random numbers, since they leave one degree of freedom for choosing $x$.

It remains to show that the protocols is oblivious and the simulators exist.

We construct the simulator $\simulator_{Participant}(x, Share_{i}(F_{k}(x))$ as follows.
The simulator outputs $2t$ random elements $r \sample \FF_p$.
This is perfectly indistinguishable, since the keyholder chooses a uniform random blinding element per message.
Let $r, r' \sample [0, |\FF_p| \cdot 2^{\lambda}]^2$.
The simulator outputs $\enc(r |\FF_p| + Share_{i}(F_{k}(x)))$ and $\enc(r' |\FF_p| + Share'_{i}(F_{k}(x)))$.
This is statistically indistinguishable, since the keyholder uses share conversion to hide the multiplications in $\FF_N$.

We construct the simulator $\simulator_{Keyholder}(k)$ as follows.
The simulator outputs $a, c \sample \FF_{\phi(p)}^2$ and programs the random oracle $H(x)$ for $x \gets \FF_p$ to $g^{c/a}$.
This is perfectly indistinguishable, since all values are uniform and the random oracle is consistent.
The simulator outputs $2t$ random elements $r \sample \FF_{q^2}$.
This is computationally indistinguishable, since Paillier ciphertexts are semantically secure~\cite{Paillier}.

\end{proof}



\section{SS-OPRF}

\subsection{Security Definition}


Requirements: 
-keyholder oblivious
-PRF output should be random (obviously)
- Given (t-1) elements all of them appear random (b/c our ss-oprf has an IT argument and a computatioal argument)
- with $H(x)^k$ and (t-1) shares still appears random 
-with t, cannot use it to solve prf property...


Obliviousness

Random PRF Output

No reconstruction even given PRF

\subsection{SS-OPRF 1}

Rasoul's construction, secret sharing in the exponent

$A -> S: x^r$ \\
$S -> A: x^{rSS(c)}$
$A: x^{SS(c)}$

Not sure I got this right.

Problem reconstruction involves modular exponentiation

\subsection{Security Proof}

\subsection{SS-OPRF 2}

My construction, 2 round protocol

$A -> S: g^r, x^r$ \\
$S -> A: (g^r)^{\log s} x^{rc}$ \\
$A -> S: E(((g^r)^{\log s} x^{rc})^{1/r}) = E(s x^c)$ \\
$S -> A: E(SS(x^c))$

\subsection{Security Proof}

\section{Completing the Protocol}

\subsection{Verifying the Reconstruction of Shares}

Give $x^c$ and $x^{cc'}$, reveal $c'$

\subsection{Distributing S}

Share $c = c_1 c_2 c_3 ...$

Also distribute decryption

\subsection{Reducing the number of possible share combinations}

Use hashing to bin the elements.  Only reconstruct within one bin

\section{Evaluation}

\subsection{SS-OPRF}

Time and communication to compute one SS-OPRF1
Time and communication to compute one SS-OPRF2

\subsection{Reconstruction}

Time to reconstruct OT-SI in the base
Time to reconstruct OT-SI in the exponent

\section{Related Work}
Private set intersection is a well-studied problem, although most
prior work deals with its standard, non-threshold formulation: the
specific case when $t=m$. We first discuss approaches to the threshold
PSI problem discussed in this paper and then give an overview of the
techniques used for standard PSI. A more comprehensive overview of
the latter is included in \cite{Pinkas}.

\subsection{Over-Threshold PSI}

Kissner and Song \cite{Kissner} proposed protocols for private set
intersection and various related problems such as cardinality set
intersection, where only the size of the set is revealed; over-threshold
intersection, which roughly corresponds to our problem; and threshold
set intersection, where it is not revealed how many parties own a
given element in the intersection. Their approach (to threshold
and over-threshold intersection) does not reveal the identities of
the parties which own a given element in the intersecting set and
involves a small amount of local computation (on the order of the
communication cost). However, their protocol's total communication
complexity is $O(nm^{3})$, which makes it less suitable than either
of our two $O(nm)$ schemes when the number of parties is large.

Threshold PSI can also be achieved using generic secure computation
protocols since it corresponds to the functionality 
\[
f(\mathbb{L}_{1},\ldots,\mathbb{L}_{m})=\{x\mid\exists i_{1},\ldots i_{t},\ x\in\cap_{r=1}^{t}\mathbb{L}_{i_{r}}\}.
\]
$f$ can be implemented using the following circuit: the union $\mathbb{L}$
of all input elements is sorted (using a sorting network), and the
resulting output is scanned for contiguous groups of at least $t$
identical elements. Note that since the required output is an (unordered)
set, we need to randomly shuffle all such found elements to avoid
leaking additional information. The communication complexity of this
approach is equal to the resulting circuit size: $O(nm\log^{2}nm)$.

\subsection{Standard PSI }

\subsubsection{Public-Key Protocols}

Many approaches to PSI rely heavily on public-key cryptography, as
in our paper. Meadows et al. \cite{Meadows} and Huberman et al. \cite{Huberman}
present protocols based on Diffie-Hellman key exchange, while several
based on RSA feature in the more recent work by De Cristofaro et al
\cite{DeCristofaro}. The protocols in \cite{Freedman2004,Freedman2016}
are similar to our work in that they make use polynomial interpolation
and the Paillier cryptosystem (or alternatively, ElGamal), ultimately
relying on the decisional Diffie-Hellman assumption. Public-key approaches
to PSI tend to have better communication complexity than other methods,
at the expense of having higher overall computational costs \cite{Pinkas}.

\subsubsection{Protocols Based on Oblivious Transfer}

Oblivious transfer (OT) \cite{Rabin} is a two-party protocol where
one of the parties, the sender, holds several pieces of data, one
of which will be sent to the receiver. The receiver can freely choose
which of the messages to learn, however, the sender does not obtain
any information about the receiver's choice. A common way to perform
a large number of OT executions with low amortized cost is through
OT extension \cite{Pinkas}, which is often used in the construction
of OPRFs. Recent work in this direction is that of Kolesnikov et al.
\cite{Kolesnikov}, who greatly improved the communication cost of protocols
based on OT extension. Further improvements were introduced by Pinkas
et al. in \cite{Pinkas} and then subsequently in \cite{SpOT}.

\section{Conclusion}

\bibliographystyle{plain}
\bibliography{references}
\end{document}
