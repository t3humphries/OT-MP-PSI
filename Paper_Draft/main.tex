\documentclass[10pt, sigconf]{acmart}
\settopmatter{printacmref=false}
\renewcommand\footnotetextcopyrightpermission[1]{} 
\usepackage[utf8]{inputenc}
\usepackage{comment}
\usepackage{multirow}
\usepackage{pifont}
\usepackage{xifthen}
\usepackage[underline=false]{pgf-umlsd}
\usepackage{tikz}
\usepackage{algorithm}
\usepackage[noend]{algpseudocode}
\usepackage{algpseudocode}
\newcommand{\formatcomment}[1]{\scriptsize\textcolor{blue!25!black}{\texttt{#1}}}
\newcommand{\InlineComment}[1]{\vspace*{0.5ex}\State\formatcomment{/*\,#1\,*/}\vspace*{0.5ex}}
\algrenewcommand{\algorithmiccomment}[1]{\hfill\parbox{5.2cm}{\formatcomment{//\,#1}}}
\algrenewcommand\algorithmicprocedure{\textbf{algorithm}}
\algnewcommand\algorithmicforeach{\textbf{for each}}
\algdef{S}[FOR]{ForEach}[1]{\algorithmicforeach\ #1\ \algorithmicdo}
\usepackage{enumitem}
\usepackage[n ,advantage, operators, sets , adversary, landau, probability, notions, logic, ff , mm, primitives, events, complexity, asymptotics, keys]{cryptocode}
\usepackage{booktabs} % For formal tables
%\usepackage{color,soul}
\usepackage{comment}
\usepackage{amssymb}
\usepackage[normalem]{ulem} % use normalem to protect \emph
 \newcommand{\superscript}[1]{\ensuremath{{}^{\textrm{\scriptsize #1}}}}
 \newcommand{\roughtext}[1]{\begin{color}{SkyBlue}#1\end{color}}
 \newcommand{\mntext}[1]{\colorbox{SkyBlue}{\begin{color}{black}#1\end{color}}}
 \newcommand{\mn}[2][]{{\tiny\superscript{\mntext{\arabic{mn}}}}\marginpar{\scriptsize{
 			\ifthenelse{\isempty{#1}}
 			{\mntext{\parbox{0.95\marginparwidth}{\superscript{\arabic{mn}}~\raggedright{#2}}}}
 			{\mntext{\parbox{0.95\marginparwidth}{\superscript{\arabic{mn}}#1 says~:~\raggedright{#2}}}}
 		}}\stepcounter{mn}}

\title{Over-Threshold Multi-Party Private Set Intersection}
\date{}
\usepackage{natbib}
\usepackage{graphicx}
\onecolumn

\begin{document}
\pagestyle{plain} % removes running headers
\onecolumn
\begingroup
\centering
{\Huge Over-Threshold Multi-Party Private Set Intersection}\\[3em]
\endgroup
% \maketitle
% \begin{abstract}
% \end{abstract}

\section{Introduction}

Use case: collaborating network operations centers under different legislations

Brief problem intro

Comparison to Kissner, Song protocol \cite{Kissner}: \\
- $O(1)$ rounds vs. $O(m)$
- problem-adaptive communication complexity $O(nmk)$ vs. $O(nm^3)$ \\

Comparison to generic secure computation: \\
- lower communication complexity $O(nmk)$ vs. $O(nm^3 \log^2 nm)$ -- assuming optimal circuit is $O(nm \log^2 nm)$ \\
- constant round multi-party protocols not yet practical

SS-OPRF

Two variants: SS in the exponent, communication O(nmk); SS in the base, communication O(nmkt)

Contributions \\
- new construction of over-threshold multi-party private set intersection protocol with communication ... and security ... \\
- two secret-shared oblivious pseudo-random function (SS-OPRF) \\
- practical evaluation?

\section{Problem Description}

$m$ parties, each with a set of size $n$, would like to compute the number of occurences of each element that occurs at least $t$, but nothing else.
The computation should be secure against a coalition of $k$ semi-honest parties.

\section{Background}

\subsection{OPRF}

$A -> S: x^a$ \\
$S -> A: x^{as}$ \\
$A: x^s$


	
\begin{definition}
	An Oblivious PseudoreRndom Function (OPRF)...
\end{definition}


\begin{definition}
	In a $(t,n)$\textit{-threshold scheme} an entity generates $n$ shares of a secret $s$ to be provided to $n$ participants such that: (1) any size-$t$ subset of the $n$ participants can compute the secret given their $t$ shares, and (2) no subset of the $n$ participants consisting of $t-1$ or fewer participants is able to gain any knowledge of the secret given their combined shares.
\end{definition}

\begin{definition}
	A Secret Sharing - Oblivious Pseudo-Random Function (SS-OPRF) is an OPRF which employs a unique secret sharing polynomial (satisfying the requirements for a threshold scheme) for each input message such that the output of the function is a set of secret shares of the pseudorandom function satisfying the following properties: 
	\begin{itemize}
		\item a
	\end{itemize} 
\end{definition}



\subsection{Paillier \cite{Paillier} and Damgard Jurik \cite{Damgard} Crytposystems}

\pagebreak
\section{Protocol Overview}
% From http://tex.stackexchange.com/questions/164707/how-to-use-greek-letters-in-pgf-umlsd-or-generally-terms-starting-with
\renewcommand{\mess}[4][0]{
  \stepcounter{seqlevel}
  \path
  (#2)+(0,-\theseqlevel*\unitfactor-0.7*\unitfactor) node (mess from) {};
  \addtocounter{seqlevel}{#1}
  \path
  (#4)+(0,-\theseqlevel*\unitfactor-0.7*\unitfactor) node (mess to) {};
  \draw[->,>=angle 60] (mess from) -- (mess to) node[midway, above]
  {#3};
  \node (\detokenize{#3} from) at (mess from) {};
  \node (\detokenize{#3} to) at (mess to) {};
}

% From http://tex.stackexchange.com/questions/98525/pgf-umlsd-and-externalize
\newcommand{\sdinit}{%
   \pgfdeclarelayer{umlsd@background}%
   \pgfdeclarelayer{umlsd@threadlayer}%
   \pgfsetlayers{umlsd@background,umlsd@threadlayer,main}%
}
\newcommand{\sdbegin}{%
   \setlength{\unitlength}{1cm}%
   \tikzstyle{sequence}=[coordinate]%
   \tikzstyle{inststyle}=[rectangle, draw, anchor=west, minimum
   height=0.8cm, minimum width=1.6cm, fill=white, 
   drop shadow={opacity=1,fill=black}]%
   \ifpgfumlsdroundedcorners%
      \tikzstyle{inststyle}+=[rounded corners=3mm]%
   \fi%
   \tikzstyle{blockstyle}=[anchor=north west]%
   \tikzstyle{blockcommentstyle}=[anchor=north west, font=\small]%
   \tikzstyle{dot}=[inner sep=0pt,fill=black,circle,minimum size=0.2pt]%
   \global\def\unitfactor{0.6}%
   \global\def\threadbias{center}%
   % reset counters
   \setcounter{preinst}{0}%
   \setcounter{instnum}{0}%
   \setcounter{threadnum}{0}%
   \setcounter{seqlevel}{0}%
   \setcounter{callevel}{0}%
   \setcounter{callselflevel}{0}%
   \setcounter{blocklevel}{0}%
   % origin
   \node[coordinate] (inst0) {};%
}
\newcommand{\sdend}{%
   \begin{pgfonlayer}{umlsd@background}%
      \ifnum\value{instnum}>0%
         \foreach \t [evaluate=\t] in {1,...,\theinstnum}{%
            \draw[dotted] (inst\t) -- ++(0,-\theseqlevel*\unitfactor-2.2*\unitfactor);%
         }%
         
      \fi%
      \ifnum\value{threadnum}>0%
         \foreach \t [evaluate=\t] in {1,...,\thethreadnum}{%
            \path (thread\t)+(0,-\theseqlevel*\unitfactor-0.1*\unitfactor) node (threadend) {};%
            \tikzstyle{threadstyle}+=[threadcolor\t]%
            \drawthread{thread\t}{threadend}%
         }%
      \fi%
   \end{pgfonlayer}%
}

\subsection{Notations and Assumptions}
\begin{itemize}
    \item m = the number of parties each identified by $id_i$; $i = 1, \cdots, m$
    \item n = max size of the set of elements, $\mathbb{L}_i$, owned by $id_i$; $i = 1, \cdots, m$
    \item t = threshold (intersection and the secret sharing threshold)
    \item b = number of bins; $b_1, \cdots, b_b$
    \item Each bin is padded to max
    \item Key holder is semi-honest
    \item Key holder is not a party
    \item Key holder sees at most t-1 shares
\end{itemize}
\subsection{Scheme 1}
Set up: field $\mathbb{F}$ with prime p, generator g, hash functions $H(\cdot)$ is used in share generation and $H_B(\cdot)$ is used for hashing-to-bins, global randomnesses: i) key holder: $K_1$ and $K_2$; $K_1$ is just known by key holder, but $k_2$ is known by all, ii) the random numbers $r_1, \cdots, r_{t-1} \gets \mathbb{F}_p$ generated by the key holder, that are fixed for all participants and their all elements. Participants use a homomorphic encryption scheme, Enc, with pubic and private keys $pk, sk$.
\subsubsection{Share Generation}\label{ShareGen_S1}
The key holder generates random numbers $r_1, \cdots, r_{t-1} \gets \mathbb{F}_p$. Then it communicates with each participant to generate shares for each participant's element using Shamir secret sharing scheme. Figure \ref{fig:S1_ShareGen} shows the protocol taking place between the key holder and a participant with identifier identifier $id_i$, $i =1, \cdots, m$, to generate shares for an element $l$. This protocol iterates over all elements $l \in \mathbb{L}$ to generate the corresponding share for each element owned by $id_i$.
\begin{figure}[h!]
   \centering
   \sdinit{}
   \begin{tikzpicture}
      % Define symbols and names for the parties
      \sdbegin{}
      \newinst{A}{Key holder}
      \newinst[5]{B}{Participant} % Increase "5" to widen
      
    % Message from B to A, with computations by both sides
      \postlevel
      \mess{B}{$H(l)^\alpha, g^\alpha $}{A}
      \node[anchor=west] at (mess from) {\shortstack[c]{
            element \emph{l}, ${id_i}$ \\
            $\alpha \gets \mathbb{F}_p$}};
      \node[anchor=east] at (mess to) {$H(l)^{K_1 \alpha}$};
      
      % Message from A to B, with precomputations
      \postlevel
      \mess{A}{$g^{r \alpha}  H(l)^{K_1 \alpha}$}{B}
      \node[anchor=east] at (mess from) {$r \gets \mathbb{F}_p$};
      \node[anchor=west] at (mess to) {$g^{r} H(l)^{K_1}$};

      \postlevel
      \mess{B}{$Enc_{pk}[g^r H(l)^{K_1}]$}{A}
      \node[anchor=west] at (mess from) {\shortstack[l]{
            Enc: homomorphic \\encryption scheme}};
      \node[anchor=east] at (mess to) {$Enc_{pk}[H(l)^{K_1}]$};

      \postlevel
      \postlevel
      \mess{A}{\shortstack[l]{$\{g^{r \alpha} H(l)^{r_j \alpha}\}_{j = 1, \cdots, t-1}$\\$\{g^{r \alpha} H(l)^{r'_j \alpha}\}_{j = 1, \cdots, t-1}$ }}{B}
      \node[anchor=east] at (mess from) {\shortstack[l]{
      For global $r_j$s\\
      And global $r'_j$s
      }};
      \node[anchor=west] at (mess to) {\shortstack[l]{
      $\{g^{r} H(l)^{r_j}\}_{j = 1, \cdots, t-1}$\\
      $\{g^{r} H(l)^{r'_j}\}_{j = 1, \cdots, t-1}$
      }}; 
      
      \postlevel
      \postlevel
      \postlevel
      \mess{B}{\shortstack[l]{$\{Enc_{pk}[g^{r}  H(l)^{r_j}]\}_{j = 1, \cdots, t-1}$\\$\{Enc_{pk}[g^{r}  H(l)^{r'_j}]\}_{j = 1, \cdots, t-1}$}}{A}
      \node[anchor=west] at (mess from) {\shortstack[l]{
            Enc: homomorphic \\encryption scheme}};
      \node[anchor=east] at (mess to) {\shortstack[l]{
      $\{Enc_{pk}[H(l)^{r_j}]\}_{j = 1, \cdots, t-1}$\\
      $\{Enc_{pk}[H(l)^{r'_j}]\}_{j = 1, \cdots, t-1}$}};

      \postlevel
      \postlevel
      \postlevel
      \mess{A}{\shortstack[c]{$P(id_i)$\\$P'(id_i)$}}{B}
      \node[anchor=east] at (mess from) {\shortstack[r]{
       $P(x) = \sum_{}x^j Enc[H(l)^{r_j}] $\\
       $ + Enc[H(l)^{K_1}]$
       \vspace{.3em}
       \\$P'(x) = \sum_{}x^j Enc[H(l)^{r'_j}] $\\
       $+ Enc[H(l)^{K_1K_2}]$}};
      \node[anchor=west] at (mess to) {\shortstack[c]{$Share_i(l) = Dec_{sk}[P(id_i)]$\\
      $MAC_i(l) = Dec_{sk}[P'(id_i)]$}};        
      \sdend{}
   \end{tikzpicture}
   \caption{Communication between the key holder and a participant $id_i$ in scheme 1 - Share generation for an element $l \in \mathbb{L}_i$, owned by $id_i$}
   \label{fig:S1_ShareGen}
\end{figure}

\subsubsection{Hashing-to-bins}\label{Binning_S1}
Participants hash their shares to bins to reduce the computation cost for the reconstructor. Corresponding shares to an element $l$ are stored in bin number $H_B(l)$. Each stored value is a quadruple consisting of i) the participant's identifier, $id_i$, ii) $id_i$'s share for its element $l$, $Share_i(l)$, iii) the corresponding MAC, $MAC_i(l)$, and iv) the number of the bin, $H_B(l)$, that stores the information.

\subsubsection{Reconstruction}\label{Recon_S1}
After each participant stored their shares in the corresponding bins, the reconstructor  $\mathcal{R}$ -- who is also a participant -- reconstructs secrets in each bin. For each $m \choose t$ subset of the participants, $id_{i_1}, \cdots, id_{i_t}$, $\mathcal{R}$ collects the shares and reconstructs the corresponding secret by applying Lagrange interpolation: $H(l)^{k_1} = \sum^t_{w=0} Share_{i_w}(\cdot)(\prod_{w' \neq w} \frac{-id_{i_{w'}}}{id_{i_w} - id_{i_{w'}}})$. Reconstruction's steps are summarized in the Algorithm \ref{alg:S1_Recon}. 

\begin{algorithm}[h!]
	 \caption[\textsc{Reconstruct\textsubscript{Scheme1}}]{\textsc{Reconstruct\textsubscript{Scheme1}}}\label{alg:S1_Recon}
	 	\begin{algorithmic}[1]
	 	    \InlineComment{$K_1$ is key holder's secret key and $K_2$ is a publicly known key used for MAC generation}
	 		\InlineComment{$H(\cdot)$ is a hash function in share generation}
	 		\InlineComment{$H_B(\cdot)$ is a hash functions used for hashing-to-bin (Section \ref{Binning_S1})}
            \ForEach {Participant $id_i$;  $i = 1, \cdots, m$}
                \ForEach {Element $l$ owned by $id_i$; $l \in \mathbb{L}_i$}
                    \State $id_i$: store $(id_i, Share_i(l), MAC_i(l), H_B(l))$ in the bin number $H_B(l)$
                    \State
	 	        \EndFor
	 	    \EndFor 
	 	    \ForEach {Bin $b_z$;  $z = 1, \cdots, b$}
	 	        \ForEach {t-subset of quadruples in the bin $b_z$; $\{(id_{i_1}, Share_{i_1}(l), MAC_{i_1}(l), b_z), \cdots, (id_{i_t}, Share_{i_t}(l), MAC_{i_t}(l), b_z)\}$}
	 	            \State $\mathcal{R}$: Apply Lagrange interpolation to  find the corresponding intercept, $Secret_{share}$, for the polynomial that covers the points $\{({i_1}, Share_{i_1}), \cdots, ({i_t}, Share_{i_t})\}$
	 	            \State $\mathcal{R}$: Apply Lagrange interpolation to  find the corresponding intercept, $Secret_{MAC}$, for the polynomial that covers the points $\{({i_1}, MAC_{i_1}), \cdots, ({i_t}, MAC_{i_t})\}$
	 	            \If {$Secret_{MAC} == (Secret_{share})^{K_2}$}{ Reveal that $(id_{i_1}, \cdots , id{i_t})$ can reconstruct the $Secret_{share}$ which is $H(l)^{K_1}$}
	 	            \EndIf 
	 	        \EndFor
	 	    \EndFor
	 	\end{algorithmic}
\end{algorithm}


\subsubsection{Complexity}
\begin{theorem}
The communication complexity of Scheme 1 is $\cdots$
% $O(nm^2)$ or $O(nmtc)$?
\end{theorem}
\begin{proof}
\end{proof}
\begin{theorem}
The computation complexity of Scheme 1 is $\cdots$
%${m \choose t} n \log{n}^m$.
\end{theorem}
\begin{proof}
\end{proof}

\subsection{Scheme 2}
In this scheme, the polynomial in Shamir secret sharing scheme is calculated in the exponent, the result is eliminating the need to use the homomorphic encryption scheme, Enc. This change reduces the communication cost of the scheme as described in Section \ref{ShareGen_S2}. Similar to Scheme 1, we have the following set up parameters: field $\mathbb{F}$ with prime p, generator g, hash functions $H(\cdot)$ is used in share generation and $H_B(\cdot)$ is used for hashing-to-bins, global randomnesses: i) key holder: $K_1$ and $K_2$; $K_1$ is just known by key holder, but $k_2$ is known by all, ii) the random numbers $r_1, \cdots, r_{t-1} \gets \mathbb{F}_p$ generated by the key holder, that are fixed for all participants and their all elements. 

\subsubsection{Share Generation}\label{ShareGen_S2}
\begin{figure}[h!]
   \centering
   \sdinit{}
   \begin{tikzpicture}
      % Define symbols and names for the parties
      \sdbegin{}
      \newinst{A}{Key holder}
      \newinst[5]{B}{Participant} % Increase "5" to widen
      
    % Message from B to A, with computations by both sides
    %   \postlevel
    %   \mess{A}{\shortstack[l]{$\{g^{r_i}\}_{i = 1, \cdots, t-1}$\\$\{g^{r'_i}\}_{i = 1, \cdots, t-1}$ }}{B}
    %   \node[anchor=east] at (mess from) {\shortstack[l]{
    %   For global $r_i$s\\
    %   And global $r'_i$s
    %   }};
    %   \node[anchor=west] at (mess to) {\shortstack[l]{
    %   $\{g^{r_i}\}_{i = 1, \cdots, t-1}$\\
    %   $\{g^{r'_i}\}_{i = 1, \cdots, t-1}$
    %   }}; 
    
      \postlevel
      \postlevel
      \mess{B}{$H(l)^\alpha$}{A}
    %   \mess{B}{\shortstack[l]{$H(l)^\alpha , \{g^{r_i \alpha}\}_{i = 1, \cdots, t-1}$ \\ $H(l)^\alpha , \{g^{r'_i \alpha}\}_{i = 1, \cdots, t-1}$}}{A}      
       \node[anchor=west] at (mess from) {\shortstack[c]{
            element \emph{l}, \emph{id} \\
            $\alpha \gets \mathbb{F}_p$}};
    \node[anchor=east] at (mess to) {\shortstack[l]{$H(l)^{\alpha K_1}, \{H(l)^{r_i \alpha}\}_{i = 1, \cdots, t-1}$ \\ $H(l)^{\alpha K_1 K_2}, \{H(l)^{r'_i \alpha}\}_{i = 1, \cdots, t-1}$}};
    %   \node[anchor=east] at (mess to) {\shortstack[l]{$H(l)^{\alpha K_1}, \{g^{r_i \alpha}\}_{i = 1, \cdots, t-1}$ \\ $H(l)^{\alpha K_1 K_2}, \{g^{r'_i \alpha}\}_{i = 1, \cdots, t-1}$}};
      
      \postlevel
      \postlevel
      \mess{A}{\shortstack[c]{$P(id)$\\$P'(id)$}}{B}
      \node[anchor=east] at (mess from) {\shortstack[r]{
       $P(x) = H(l)^{\alpha \sum_{}r_i x^i} H(l)^{\alpha K_1}$
       \vspace{.4em}
       \\$P'(x) = H(l)^{\alpha \sum_{}r'_i x^i} H(l)^{\alpha K_1 K_2}$}};
      \node[anchor=west] at (mess to) {\shortstack[c]{$Share(l) = P(id)^{1/ \alpha}$\\
      $MAC(l) = P'(id)^{1/ \alpha}$}};        
      
      \sdend{}
   \end{tikzpicture}
   \caption{Communication between the key holder and a participant $id_i$ in scheme 2 - Share generation for an element $l \in \mathbb{L}_i$, owned by $id_i$}
   \label{fig:S2_ShareGen}
\end{figure}
The key holder generates random numbers $r_1, \cdots, r_{t-1} \gets \mathbb{F}_p$. Then it communicates with each participant to generate shares for each participant's element using Shamir secret sharing scheme, by forming a polynomial in the exponent. Figure \ref{fig:S2_ShareGen} shows the protocol taking place between the key holder and a participant with identifier identifier $id_i$, $i =1, \cdots, m$, to generate shares for an element $l$. This protocol iterates over all elements $l \in \mathbb{L}$ to generate the corresponding share for each element owned by $id_i$.

\subsubsection{Hashing-to-bins}\label{Binning_S2} This step is identical to the hashing-to-bins described in Section \ref{Binning_S1}.

\subsubsection{Reconstruction}\label{Recon_S2}
Similar to the reconstruction in Section \ref{Recon_S1} for Scheme 1, each participant stores their shares in the corresponding bins. Then the reconstructor $\mathcal{R}$ --who is also participant-- reconstructs the secrets in each bin for every $m \choose t$ subset of the participants, $id_{i_1}, \cdots, id_{i_t}$, $\mathcal{R}$. As the polynomial is in the exponent of the generator in this scheme, the reconstructor applies the Lagrange interpolation in the exponent to calculate the secret as follows: $H(l)^{k_1} = \prod^t_{w=0} Share_{i_w}(\cdot)^{(\prod_{w' \neq w} \frac{-id_{w'}}{id_w - id_{w'}})}$.

\begin{algorithm}[h!]
	 \caption[\textsc{Reconstruct\textsubscript{Scheme2}}]{\textsc{Reconstruct\textsubscript{Scheme2}}}\label{alg:S2_Recon}
	 	\begin{algorithmic}[1]
	 	    \InlineComment{$K_1$ is key holder's secret key and $K_2$ is a publicly known key used for MAC generation}
	 		\InlineComment{$H(\cdot)$ is a hash function in share generation}
	 		\InlineComment{$H_B(\cdot)$ is a hash functions used for hashing-to-bin (Section \ref{Binning_S1})}
            \ForEach {Participant $id_i$;  $i = 1, \cdots, m$}
                \ForEach {Element $l$ owned by $id_i$; $l \in \mathbb{L}_i$}
                    \State $id_i$: store $(id_i, Share_i(l), MAC_i(l), H_B(l))$ in the bin number $H_B(l)$
                    \State
	 	        \EndFor
	 	    \EndFor 
	 	    \ForEach {Bin $b_z$;  $z = 1, \cdots, b$}
	 	        \ForEach {t-subset of quadruples in the bin $b_z$; $\{(id_{i_1}, Share_{i_1}(l), MAC_{i_1}(l), b_z), \cdots, (id_{i_t}, Share_{i_t}(l), MAC_{i_t}(l), b_z)\}$}
	 	            \State $\mathcal{R}$: Reconstruct the corresponding secret, $Secret_{share}$, to the points $\{({i_1}, Share_{i_1}), \cdots, ({i_t}, Share_{i_t})\}$
	 	            \State $\mathcal{R}$: Reconstruct the corresponding secret, $Secret_{MAC}$, to the points $\{({i_1}, MAC_{i_1}), \cdots, ({i_t}, MAC_{i_t})\}$
	 	            \If {$Secret_{MAC} == (Secret_{share})^{K_2}$}{ Reveal that $(id_{i_1}, \cdots , id{i_t})$ can reconstruct the $Secret_{share}$ which is $H(l)^{K_1}$}
	 	            \EndIf 
	 	        \EndFor
	 	    \EndFor
	 	\end{algorithmic}
\end{algorithm}

\subsubsection{Complexity}
\begin{theorem}
The communication complexity of Scheme 2 is $\cdots$.
\end{theorem}
\begin{proof}
\end{proof}
\begin{theorem}
The computation complexity of Scheme 2 is $\cdots$.
\end{theorem}
\begin{proof}
\end{proof}


\pagebreak


\section{SS-OPRF}

\subsection{Security Definition}


Requirements: 
-keyholder oblivious
-PRF output should be random (obviously)
- Given (t-1) elements all of them appear random (b/c our ss-oprf has an IT argument and a computatioal argument)
- with $H(x)^k$ and (t-1) shares still appears random 
-with t, cannot use it to solve prf property...


Obliviousness

Random PRF Output

No reconstruction even given PRF

\subsection{SS-OPRF 1}

Rasoul's construction, secret sharing in the exponent

$A -> S: x^r$ \\
$S -> A: x^{rSS(c)}$
$A: x^{SS(c)}$

Not sure I got this right.

Problem reconstruction involves modular exponentiation

\subsection{Security Proof}

\subsection{SS-OPRF 2}

My construction, 2 round protocol

$A -> S: g^r, x^r$ \\
$S -> A: (g^r)^{\log s} x^{rc}$ \\
$A -> S: E(((g^r)^{\log s} x^{rc})^{1/r}) = E(s x^c)$ \\
$S -> A: E(SS(x^c))$

\subsection{Security Proof}

\section{Completing the Protocol}

\subsection{Verifying the Reconstruction of Shares}

Give $x^c$ and $x^{cc'}$, reveal $c'$

\subsection{Distributing S}

Share $c = c_1 c_2 c_3 ...$

Also distribute decryption

\subsection{Reducing the number of possible share combinations}

Use hashing to bin the elements.  Only reconstruct within one bin

\section{Evaluation}

\subsection{SS-OPRF}

Time and communication to compute one SS-OPRF1
Time and communication to compute one SS-OPRF2

\subsection{Reconstruction}

Time to reconstruct OT-SI in the base
Time to reconstruct OT-SI in the exponent

\section{Related Work}

Kissner, Song

Generic Secure Computation, PSI via Secure computation

PSI protocols based on OPRF

PSI protocols based on hashing




Efficient Batched Oblivious PRF with Applications to Private Set Intersection, CCS 2016 Vladimir Kolesnikov, Ranjit  Kumaresan, Mike Rosulek, Ni Trieu





\section{Conclusion}

\bibliographystyle{plain}
\bibliography{references}
\end{document}
