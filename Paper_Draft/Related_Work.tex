Private set intersection is a well-studied problem, although most
prior work deals with its standard, non-threshold formulation: the
specific case when $t=m$. We first discuss approaches to the threshold
PSI problem discussed in this paper and then give an overview of the
techniques used for standard PSI. A more comprehensive overview of
the latter is included in \cite{Pinkas}.

\subsection{Over-Threshold PSI}

Kissner and Song \cite{Kissner} proposed protocols for private set
intersection and various related problems such as cardinality set
intersection, where only the size of the set is revealed; over-threshold
intersection, which roughly corresponds to our problem; and threshold
set intersection, where it is not revealed how many parties own a
given element in the intersection. Their approach (to threshold
and over-threshold intersection) does not reveal the identities of
the parties which own a given element in the intersecting set and
involves a small amount of local computation (on the order of the
communication cost). However, their protocol's total communication
complexity is $O(nm^{3})$ and takes $O(m)$ rounds, which makes it less suitable
than our $O(nmt)$ scheme requiring $O(1)$ rounds.

Threshold PSI can also be achieved using generic secure computation
protocols since it corresponds to the functionality 
\[
f(\mathbb{L}_{1},\ldots,\mathbb{L}_{m})=\{x\mid\exists i_{1},\ldots i_{t},\ x\in\cap_{r=1}^{t}\mathbb{L}_{i_{r}}\}.
\]
$f$ can be implemented using the following circuit: the union $\mathbb{L}$
of all input elements is sorted (using a sorting network), and the
resulting output is scanned for contiguous groups of at least $t$
identical elements. Note that since the required output is an (unordered)
set, we need to randomly shuffle all such found elements to avoid
leaking additional information. The communication complexity of this
approach depends on the resulting circuit size.  The circuit size for 
sorting (and random shuffling) is $O(nm\log^{2}nm)$. The circuit size for
scanning depends whether it is done sequentially with multiplicative depth
$O(nm)$ in size $O(nm)$ or $O(\log t)$ multiplicative depth in size $O(nmt)$.

\subsection{Standard PSI }

\subsubsection{Public-Key Protocols}

Many approaches to PSI rely heavily on public-key cryptography, as
in our paper. Meadows et al. \cite{Meadows} and Huberman et al. \cite{Huberman}
present protocols based on Diffie-Hellman key exchange, while several
based on RSA feature in the more recent work by De Cristofaro et al
\cite{DeCristofaro}. The protocols in \cite{Freedman2004,Freedman2016}
are similar to our work in that they make use polynomial interpolation
and the Paillier cryptosystem (or alternatively, ElGamal), ultimately
relying on the decisional Diffie-Hellman assumption. Public-key approaches
to PSI tend to have better communication complexity than other methods,
at the expense of having higher overall computational costs \cite{Pinkas}.

\subsubsection{Protocols Based on Oblivious Transfer}

Oblivious transfer (OT) \cite{Rabin} is a two-party protocol where
one of the parties, the sender, holds several pieces of data, one
of which will be sent to the receiver. The receiver can freely choose
which of the messages to learn, however, the sender does not obtain
any information about the receiver's choice. A common way to perform
a large number of OT executions with low amortized cost is through
OT extension \cite{Pinkas}, which is often used in the construction
of OPRFs. Recent work in this direction is that of Kolesnikov et al.
\cite{Kolesnikov}, who greatly improved the communication cost of protocols
based on OT extension. Further improvements were introduced by Pinkas
et al. in \cite{Pinkas} and then subsequently in \cite{SpOT}.