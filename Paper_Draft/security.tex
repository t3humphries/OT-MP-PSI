\begin{definition}
We say the Decisional Diffie-Hellman (DDH) Assumptions holds if
$$
\pseudocode{
a, b, c \sample \FF_{\phi(p)}^3 \\
\prob{\adv(g^a, g^b, g^c) = 1} - \prob{\adv(g^a, g^b, g^{ab}) = 1} < \negl \\
}
$$
\end{definition}

\begin{theorem}
\label{thm:sec_prf}
If the DDH Assumption holds and $H$\bmargin{*}{which H. The hash function $H(\cdot)$ or need a different notation here } is a programmable random oracle, the output of $t$ participants in Scheme 2 is a pseudo-random function.
\end{theorem}

\begin{proof}
We prove by reducing an adversary in $\pcgamename^{\prf}$ to an adversary against the DDH Assumption.
Let $g^a, g^b, g^c$ be a DDH instance.
We program $H$ to output $g^a$ on input $x$.
We set the public $\pk$ to $g^b$.
We can still answer queries for $F_{k}(x')$ where $x' \neq x$ by programming $H$ to choose and store $r \sample \FF_{\phi(p)}$ and output $R = g^r$.
An answer to an OPRF query for $x'$ is then $g^{b r} = H(x')^b$.
We set $y_b$ to $g^c$.
We set the output of adversary $\adv^{\mathsf{DDH}}$ to the output of adversary $\adv^{\prf}$.
All outputs are indistinguishable from the real scheme and any advantage between the two adversaries translates directly.
\end{proof}

\begin{theorem}
Scheme 2 is a secret-shared oblivious pseudo-random function.
\end{theorem}

\begin{corollary}
There exist simulators $\simulator_{Participant}(x, Share_{i}(F_{k}(x))$ and $\simulator_{Keyholder}(k)$ that are computationally indistinguishable from the messages received during the protocol.
\end{corollary}

\begin{proof}
Correctness is given by the protocol construction, such that $Share_{i}(F_{k}(x))$ is returned to the participant and $t' \leq t$ participants\bmargin{*}{I'm confused by the leq reconstruction here} can reconstruct $F_k(x)$.

Since each coefficient of the Shamir secret shares is the output of a PRF (Theorem~\ref{thm:ec_prf}), the output of Scheme 2 is computationally indistinguishable from a secret share of $F_{k}(x)$.
Furthermore, $t-1$ secret shares of $F_{k}(x)$ are perfectly indistinguishable from a set of $t-1$ uniformly chosen random numbers, since they leave one degree of freedom for choosing $x$.

It remains to show that the protocols is oblivious and the simulators exist.

We construct the simulator $\simulator_{Participant}(x, Share_{i}(F_{k}(x))$ as follows.
The simulator outputs $2t$ random elements $r \sample \FF_p$.
This is perfectly indistinguishable, since the keyholder chooses a uniform random blinding element per message.
Let $r, r' \sample [0, |\FF_p| \cdot 2^{\lambda}]^2$.
The simulator outputs $\enc(r |\FF_p| + Share_{i}(F_{k}(x)))$ and $\enc(r' |\FF_p| + Share'_{i}(F_{k}(x)))$.
This is statistically indistinguishable, since the keyholder uses share conversion to hide the multiplications in $\FF_N$.

We construct the simulator $\simulator_{Keyholder}(k)$ as follows.
The simulator outputs $a, c \sample \FF_{\phi(p)}^2$ and programs the random oracle $H(x)$ for $x \gets \FF_p$ to $g^{c/a}$.
This is perfectly indistinguishable, since all values are uniform and the random oracle is consistent.
The simulator outputs $2t$ random elements $r \sample \FF_{q^2}$.
This is computationally indistinguishable, since Paillier ciphertexts are semantically secure~\cite{Paillier}.

\end{proof}
