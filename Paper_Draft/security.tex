\begin{definition}
We say the Decisional Diffie-Hellman (DDH) Assumptions holds if for any PPT adversary $\adv$
$$
\pseudocode{
a, b, c \sample \ZZ_{q} \\
\prob{\adv(g^a, g^b, g^c) = 1} - \prob{\adv(g^a, g^b, g^{ab}) = 1} < \negl \\
}
$$
\end{definition}

\begin{theorem}
\label{thm:sec_prf}
If the DDH Assumption holds and $H(\cdot)$\bmargin{*}{which H. The hash function $H(\cdot)$ or need a different notation here }\fmargin{+}{Fixed.} is a programmable random oracle, each coefficient of the share share's polynomial in Scheme 2 is a pseudo-random function.
\end{theorem}

\begin{proof}
We prove by reducing an adversary in $\pcgamename^{\prf}$ to an adversary against the DDH Assumption.
Let $g^a, g^b, g^c$ be a DDH instance.
We program $H(\cdot)$ to output $g^a$ on input $x$.
We set the public $\pk$ to $g^b$.
We can still answer queries for $F_{k}(x')$ where $x' \neq x$ by programming $H(\cdot)$ to choose and store $r \sample \ZZ_{q}$ and output $R = g^r$.
An answer to an OPRF query for $x'$ is then $g^{b r} = H(x')^b$.
We set $y_b$ to $g^c$.
We set the output of adversary $\adv^{\mathsf{DDH}}$ to the output of adversary $\adv^{\prf}$.
All outputs are indistinguishable from the real scheme and any advantage between the two adversaries translates directly.
\end{proof}

\begin{theorem}
Scheme 2 is a secret-shared oblivious pseudo-random function.
\end{theorem}

\begin{corollary}
There exist simulators $\simulator_{Participant}(x, Share_{i}(x)$ and $\simulator_{Keyholder}(k)$ that are computationally indistinguishable from the messages received during the protocol.
\end{corollary}

\begin{proof}
Correctness is given by the protocol construction, such that $Share_{i}(x)$ is returned to the participant and $t' \geq t$ participants\bmargin{*}{I'm confused by the leq reconstruction here}\fmargin{+}{Fixed.} can reconstruct $0$.

Since each coefficient of the Shamir secret shares is the output of a PRF (Theorem~\ref{thm:sec_prf}), the output of Scheme 2 is computationally indistinguishable from a secret share of $0$.
Furthermore, $t-1$ secret shares are perfectly indistinguishable from a set of $t-1$ uniformly chosen random numbers, since they leave one degree of freedom for choosing $x$, as long as it is not a priori known that they reconstruct to $0$.
To explain further, consider an adversary that controls $t-2$ parties.
If this adversary obtains a share $Share_{i}(x')$ where party $i$ is not controlled by the adversary, e.g. during reconstruction.
Then this adversary cannot determine whether $x' = x$ for any x chosen by the adversary, since any $t-1$ shares may reconstruct to $0$.
Now, consider an adversary that controls $t-1$ parties.
This adversary can choose $x$, obtain secret shares and reconstruct the coefficients of the secret share polynomial using $t-1$, since it knows the ``secret'' $0$.
Hence, it can test whether for another share $Share_{i}(x')$ it holds $x'= x$.
However, the output of this test is included in the output of the protocol and the attack would be feasible for any adversary admissible to the protocol.
We repeat, $t-1$ secret shares are perfectly indistinguishable from a set of $t-1$ uniformly chosen random numbers, as long as it is not a priori known that they reconstruct to $0$.

It remains to show that the protocols is oblivious and the simulators exist.

We construct the simulator $\simulator_{Participant}(x, Share_{i}(x)$ as follows.
The simulator outputs $2t$ random elements $r \sample \FF_p$.
This is perfectly indistinguishable, since the keyholder chooses a uniform random blinding element per message.
Let $r, r' \sample [0, 2^{\lambda} p]$.
The simulator outputs $\enc(r p + Share_{i}(x))$ and $\enc(r' p + Share'_{i}(x))$.
This is statistically indistinguishable, since the keyholder uses share conversion to hide the multiplications in $\FF_N$.

We construct the simulator $\simulator_{Keyholder}(k)$ as follows.
The simulator outputs $a, c \sample \ZZ_{q}$ and programs the random oracle $H(\cdot)$ for $x \gets mathbb{L}$ to $g^{c/a}$.
This is perfectly indistinguishable, since all values are uniform and the random oracle is consistent.
The simulator outputs $2t$ random elements $r \sample \FF_{N^2}$.
This is computationally indistinguishable, since Paillier ciphertexts are semantically secure~\cite{Paillier}.

\end{proof}
