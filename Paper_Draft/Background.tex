\subsection{Secret Sharing}

In secret sharing, the goal is to distribute among $m$ parties \emph{shares
}$s_{1},\ldots,s_{m}$ of a secret $s$. A \emph{$(t,m)$-threshold
scheme} meets the conditions that any size $t$ subset of the shares is sufficient to \emph{reconstruct
}$s$ but no group of parties possessing fewer than $t$ shares can
infer any information about $s$. In Shamir's secret sharing scheme
\cite{Shamir}, the distributing party forms the polynomial 
\[
f(x)=c_{t-1}x^{t-1}+c_{t-2}x^{t-2}+\ldots c_{1}x+s
\]
over a finite field $\FF_p$ of prime order $p$, with $c_{1},\ldots, c_{t-1}$ randomly
generated. It then produces the shares by evaluating $f$ at $m$
publicly-known distinct values: for instance, by setting $s_{id_i}=(id_i,f(id_i))$.
Since $t$ points uniquely determine a polynomial of degree $t$,
given all the $s_{id_i}$, anyone possessing $t$ shares can use Lagrange
interpolation to recover $f$ and thus obtain $s$. However, fewer
than $t$ shares together reveal no information about $s$.

\subsection{Paillier Cryptosystem}

The Paillier cryptosystem \cite{Paillier} is an additively homomorphic
scheme for public key encryption based on the intractability of the
Composite Residuosity Class Problem. Ciphertexts are regarded as elements
over the multiplicative field $\FF_{N^2} = \mathbb{Z}_{N^{2}}^{*}$, where $N=p'q'$
for primes $p'$ and $q'$. Letting $\mu$ be the least common multiple of $p-1$ and $q-1$,
and choosing $g$ to be a random base such that its order is divisible
by $N$, the public key is $(N,g)$ and the private key is $(p,q)$.
Encrypting a plaintext message $0\leq x<N$ is done by choosing a random
$r<N$ and computing: 
\[
Enc(x)=g^{x}r^{N}\mod N^{2}.
\]
A given ciphertext $C<N^{2}$ is decrypted using $\mu$: 
\[
Dec(C)=\frac{L(C^{\mu}\text{ mod }N^{2})}{L(g^{\mu}\text{ mod }N^{2})}\mod N,
\]
where 
\[
L(x)=\frac{x-1}{N}.
\]
Such a scheme satisfies the following homomorphic properties (the
moduli are implicit): 
\begin{align*}
Dec(Enc(x_{1})Enc(x_{2})) & =x_{1}+x_{2}\\
Dec(Enc(x_{1})^{x_{1}}) & =x_{1}x_{2}.
\end{align*}

%D\aa mgard Jurik \cite{Damgard} 

\subsection{Oblivious Pseudo-Random Function}

Let $F_{k}(x)$ be a pseudo-random function (PRF) with key $k$ on input $x$.
We say a pseudo-random function is secure if $\advantage{\pcgamename^{\prf}}{\adv} < \frac{1}{2} + \negl$

$$
\procedure{$\pcgamename^{\prf}$}{
(k, \pk) \gets \kgen(\secparam) \\
( \cryptost, x) \gets \adv(\pk) \\
y_0 \gets F_{k}(x) \\
y_1 \sample \GG \\
b \sample {0,1} \\
b' \gets \adv(y_b, \cryptost) \\
\pcreturn b = b' }
$$

An oblivious pseudo-random function (OPRF) \cite{OPRF,Pinkas} is
a two-party protocol where the client can query a pseudo-random function
$F_{k}$ on input values $x$, in such a way that the key $k$, generated
by the server, remains hidden from the client, while at the same time
the client's input $x$ is not revealed to the server. OPRFs can be
established through generic methods for secure multiparty computation
(on top of circuits that implement ordinary pseudo-random functions),
or by means of the Diffie-Hellman assumption over a group $\GG_q$.
%For instance, the client
%chooses a random $r$ and sends $x^{r}$ to the server; the server
%responds by sending back $x^{kr}$, from which the client can obtain
%$x^{k}=(x^{kr})^{-r}$.

An oblivious pseudo-random function (OPRF) is defined as a protocol between a keyholder and client, such that
\begin{itemize}
\item the client inputs $x$ and outputs $F_{k}(x)$ and nothing else.
\item the keyholder inputs $k$ and outputs nothing.
\end{itemize}

A secret-shared oblivious pseudo-random function (SS-OPRF) is defined as a pairwise protocol between a keyholder and one of $m$ clients, such that
\begin{itemize}
\item the client inputs $x$.
\item any subset of clients of size $t - 1$ or smaller outputs nothing.
\item any subset of clients of size $t$ or larger outputs $F_{k}(x)$ and nothing else.
\item the keyholder inputs a key $k$ and outputs nothing.
\end{itemize}


